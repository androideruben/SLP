%%%%%%%%%%%%%%%%%%%%%%%%%%%%%%%%%%%%%%%%%%%%%%%%%%%%%%%%%%%%%%%%%%%%%%%%%%%%%%%%%%%%%%%%%%%%%%%%%%%%%%%%%%%%%%%%%%%%%%%%%%%%%%%%%%%
%https://nw360.blogspot.com/2017/03/set-up-pdf-viewer-in-texniccenter.html
%%%%%%%%%%%%%%%%%%%%%%%%%%%%%%%%%%%%%%%%%%%%%%%%%%%%%%%%%%%%%%%%%%%%%%%%%%%%%%%%%%%%%%%%%%%%%%%%%%%%%%%%%%%%%%%%%%%%%%%%%%%%%%%%%%%
\documentclass[letterpaper,11pt]{article}
\usepackage{float}
\usepackage{graphicx}
\usepackage{amsmath}
\usepackage{amssymb}
\usepackage{bm}         
\usepackage{amsthm}
\usepackage{afterpage}
\usepackage[table, usenames, dvipsnames]{xcolor}
\usepackage{latexsym}

%\usepackage[framemethod=0,ntheorem]{mdframed}

\usepackage{etexcmds}
\usepackage{fullpage}
\usepackage{setspace}
\usepackage{multirow}
\usepackage{rotating}
\usepackage{pdflscape}
\usepackage{fancyhdr}
\usepackage[utf8]{inputenc}
\usepackage{lscape}
\usepackage{color}
\usepackage{tabularx}
\usepackage{array}
\usepackage{longtable}
\usepackage[english]{babel}
\usepackage{csquotes}

\usepackage[font=small,labelfont=bf,skip=0pt]{caption}
%\captionsetup[table]{skip=0.2pt}
%\captionsetup[table]{aboveskip=0pt}
%\captionsetup[table]{belowskip=-15pt}

\usepackage{subcaption}

%\captionsetup[subtable]{skip=2pt}

\usepackage[tableposition=top]{caption}
\usepackage[toc,page]{appendix}
\usepackage{amsfonts}

\setcounter{MaxMatrixCols}{30}
\providecommand{\U}[1]{\protect\rule{.1in}{.1in}}

\newcommand{\tablefont}{\fontsize{3mm}{3mm}\selectfont}
\MakeOuterQuote{"}

\newcommand{\MC}{\multicolumn}
\newcommand{\MR}{\multirow}

\headheight 15pt
\headsep 2em

\renewcommand{\footrulewidth}{0pt}
\newtheorem{remark}{Remark}
\newtheorem{comment}{\textbf{Comment}}
\newtheorem{definition}{Definition}
\newtheorem{discussion}{Discussion:}
\newtheorem{claim}{Claim}
\newtheorem{question}{Question}
\newtheorem{answer}{Answer}
\newcolumntype{T}{>{\tiny}l}
\newcolumntype{M}{>{\centering\arraybackslash}m{3.1cm}}

\usepackage{fancyhdr}
\fancyhf{}
\pagestyle{fancy} %added by CDE to allow headers, footers to work
%\rhead{Share\LaTeX}
%\lhead{Guides and tutorials}


\usepackage[pdftex]{hyperref}   
\hypersetup{colorlinks, citecolor=Violet, linkcolor=Mahogany, urlcolor=blue}

\rfoot{Page \thepage}
%\lfoot{DRAFT-DELIBERATIVE-CONFIDENTIAL}  

%%%%%%%%%%%%%%%%%%%%%%%%%%%%%%%%%%%%%%%%%%%%%%%%%%%%%%%%%%%%%%%%%%%%%%%%%%%%%%%%%%%%%%%%%%%%%%%%%%%%%%%%%%%%%%
\begin{document}
%%%%%%%%%%%%%%%%%%%%%%%%%%%%%%%%%%%%%%%%%%%%%%%%%%%%%%%%%%%%%%%%%%%%%%%%%%%%%%%%%%%%%%%%%%%%%%%%%%%%%%%%%%%%%%

\vspace{-25pt}%

\begin{tabular}[t]{lp{1in}l}
	\multirow{5}{*}{\includegraphics[width = 0.9in]{UASLP.png}} \\
																								&& UASLP \\
																								&& \\

																								&& Facultad de Agronom\'ia y Agricultura. \\
																								&& \\
																								&& \\

																								&& Soluci\'on al examen 2 de Estad\'istica. \\
																								&& \\
																								&& Programaci\'on en R, \\ 
																								&& Regresi\'on Lineal. \\
																								&& \\
																								&& \\
\typeout{************RUBEN IS HERE DEBBUGING THE LOG}

\end {tabular}
\newpage
\noindent 


%%%%%%%%%%%%%%%%%%%%%%%%%%%%%%%%%%%%%%%%%%%%%%%%%%%%%%%%%%%%%%%%%%%%%%%%%%%%%%%%%%%%%%%%%%%%%%%%%%%%%%%%%%%%%

\section*{\textbf{\textit{Soluciones}}}

\section{COMPARACION DE DOS MODELOS}

\begin{itemize}
\item \textbf{\textit{Reporta por escrito en papel}} el modelo estimado de la base de datos hombres y el 
modelo estimado de la base de datos mujeres 

\\

\textbf{\textit{Soluci\'on:}}

Modelo hombres: TiempoLog= -2.823196+ 1.112214* DistanciaLog

\\

Modelo mujeres: TiempoLog= -2.69216+ 1.11167* DistanciaLog


\item \textbf{\textit{Reporta por escrito en papel}} cual de los dos modelos indica m\'as rapidez

\textbf{\textit{Soluci\'on:}} Rapidez es la pendiente del modelo, la pendiente del modelo de hombres es ligeramente mayor al del modelo de mujeres, entonces, el modelo de hombres tiene mayor rapidez.

\item \textbf{\textit{Reporta por escrito en papel}} las hip\'otesis nula $H_{0}$ y alternativa $H_{A}$ de 
la intercept y slope(s) de cada uno de los dos modelos y tu decisi\'on de aceptar o no las hip\'otesis nula con un nivel de 
significancia de $\alpha$= 0.05 (usa los p-value). Reporta tambi\'en si las relaciones son crecientes o no 
con base en cada pendiente

$H_{0}$: Intercept=0

$H_{A}$: Intercept \neq 0

Rechazar la hip\'otesis nula porque los p-values son menored a 0.05

Similarmente para las hip\'otesis de los coeficientes de cada modelo: rechazar las hip\'otesis nulas de que las pendientes son cero.

Cada pendiente en cada modelo es positiva, por lo tanto las relaciones son crecientes (a mayor distancia, mayort tiempo)

\item \textbf{\textit{Reporta por escrito en papel}} el modelo con los datos concatenados, pero
solamente los par\'ametros (ya sea intercept, $\beta$ o ambos) significativos al nivel $\alpha$= 0.05

Modelo de datos concatenados, con sexo una dummy variable e interacciones sexo*distancia, solo coeficientes (pendientes) significativas
(significativamente distintas de cero):\\

\textbf{\textit{Soluci\'on:}} TiempoLog= -2.6921619+ 1.1116747* DistanciaLog


\item \textbf{\textit{Reporta por escrito en papel}} si la interacci\'on Sexo*log(Distancia) es significativa.
(Nota: esto se interpreta como los efectos de sexo con log(Distancia) no son significativos y podemos 
suponer que la rapidez es pararlela entre hombres y mujeres y ANCOVA es un buen modelo)

\textbf{\textit{Soluci\'on:}} La prueba de hip\'otesis de la interacci\'on DistanciaLog*Sexo es de que la pendiente es distinta de cero.
Como el p-value es 0.983, que es mayor a 0.05, no se rechaza la hip\'otesis nula y se concluye que la pendiente es cero (a pesar de que
num\'ericamente la pendiente es 0.0005397, esto es, estad\'isticamente la pendiente es cero, pero num\'ericamente no lo es).

\item \textbf{\textit{Reporta por escrito en papel}} tu prediccion del tiempo que los hombres correr\'ian la Distancia de 60
y el tiempo que las mujeres correr\'ian la Distancia 200. Ten cuidado, nota que tu modelo est\'a en la escala log,
as\'i que calcular\'as con calculadora el log(Distancia de la pregunta) y luego tendr\'as que usar exponencial
porque log es base e.

\textbf{\textit{Soluci\'on:}} a. hombres corren la distancia 60 usando el modelo de hombres en 5.64399 y se resuelve as\'i: \\
TiempoLog= -2.823196+ 1.112214* log(60) \\
exp(TiempoLog) es 5.64399

(Ve el resto de posibilidades en rm examen2 solucion knit R.pdf o en rm examen2 solucion SAS output.pdf).

\item \textbf{\textit{Reporta por escrito en papel}} la \textit{Multiple R-squared}, esta la encuentras al 
ejecutar \textit{anova} (o en la tabla de resultados en SAS). Este n\'umero es deseable que este cerca de 1.00 porque en porcentaje explica 
que tan bien la variaci\'on de Y est\'a explicada por el modelo
\end{itemize}

\textbf{\textit{Soluci\'on:}} Para el modelo de hombres, la R-Squared es 0.9995
Localiza el resto de las R-squared en \textit{rm examen2 solucion knit R.pdf} o en \textit{rm examen2 solucion SAS output.pdf}.

%%%%%%%%%%%%%%%%%%%%%%%%%%%%%%%%%%%%%%%%%%%%%%%%%%%%%%%%%%%%%%%%%%%%%%%%%%%%%%%%%%%%%%%%%%%%%%%%%%%%%%%%%%%%%%
\end{document} 
%%%%%%%%%%%%%%%%%%%%%%%%%%%%%%%%%%%%%%%%%%%%%%%%%%%%%%%%%%%%%%%%%%%%%%%%%%%%%%%%%%%%%%%%%%%%%%%%%%%%%%%%%%%%%%
