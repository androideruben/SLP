%%%%%%%%%%%%%%%%%%%%%%%%%%%%%%%%%%%%%%%%%%%%%%%%%%%%%%%%%%%%%%%%%%%%%%%%%%%%%%%%%%%%%%%%%%%%%%%%%%%%%%%%%%%%%%%%%%%%%%%%%%%%%%%%%%%
%https://nw360.blogspot.com/2017/03/set-up-pdf-viewer-in-texniccenter.html
%%%%%%%%%%%%%%%%%%%%%%%%%%%%%%%%%%%%%%%%%%%%%%%%%%%%%%%%%%%%%%%%%%%%%%%%%%%%%%%%%%%%%%%%%%%%%%%%%%%%%%%%%%%%%%%%%%%%%%%%%%%%%%%%%%%
\documentclass[letterpaper,11pt]{article}
\usepackage{float}
\usepackage{graphicx}
\usepackage{amsmath}
\usepackage{amssymb}
\usepackage{bm}         
\usepackage{amsthm}
\usepackage{afterpage}
\usepackage[table, usenames, dvipsnames]{xcolor}
\usepackage{latexsym}

%\usepackage[framemethod=0,ntheorem]{mdframed}

\usepackage{etexcmds}
\usepackage{fullpage}
\usepackage{setspace}
\usepackage{multirow}
\usepackage{rotating}
\usepackage{pdflscape}
\usepackage{fancyhdr}
\usepackage[utf8]{inputenc}
\usepackage{lscape}
\usepackage{color}
\usepackage{tabularx}
\usepackage{array}
\usepackage{longtable}
\usepackage[english]{babel}
\usepackage{csquotes}

\usepackage[font=small,labelfont=bf,skip=0pt]{caption}
%\captionsetup[table]{skip=0.2pt}
%\captionsetup[table]{aboveskip=0pt}
%\captionsetup[table]{belowskip=-15pt}

\usepackage{subcaption}

%\captionsetup[subtable]{skip=2pt}

\usepackage[tableposition=top]{caption}
\usepackage[toc,page]{appendix}
\usepackage{amsfonts}

\setcounter{MaxMatrixCols}{30}
\providecommand{\U}[1]{\protect\rule{.1in}{.1in}}

\newcommand{\tablefont}{\fontsize{3mm}{3mm}\selectfont}
\MakeOuterQuote{"}

\newcommand{\MC}{\multicolumn}
\newcommand{\MR}{\multirow}

\headheight 15pt
\headsep 2em

\renewcommand{\footrulewidth}{0pt}
\newtheorem{remark}{Remark}
\newtheorem{comment}{\textbf{Comment}}
\newtheorem{definition}{Definition}
\newtheorem{discussion}{Discussion:}
\newtheorem{claim}{Claim}
\newtheorem{question}{Question}
\newtheorem{answer}{Answer}
\newcolumntype{T}{>{\tiny}l}
\newcolumntype{M}{>{\centering\arraybackslash}m{3.1cm}}

\usepackage{fancyhdr}
\fancyhf{}
\pagestyle{fancy} %added by CDE to allow headers, footers to work
%\rhead{Share\LaTeX}
%\lhead{Guides and tutorials}


\usepackage[pdftex]{hyperref}   
\hypersetup{colorlinks, citecolor=Violet, linkcolor=Mahogany, urlcolor=blue}

\rfoot{Page \thepage}
%\lfoot{DRAFT-DELIBERATIVE-CONFIDENTIAL}  

%%%%%%%%%%%%%%%%%%%%%%%%%%%%%%%%%%%%%%%%%%%%%%%%%%%%%%%%%%%%%%%%%%%%%%%%%%%%%%%%%%%%%%%%%%%%%%%%%%%%%%%%%%%%%%
\begin{document}
%%%%%%%%%%%%%%%%%%%%%%%%%%%%%%%%%%%%%%%%%%%%%%%%%%%%%%%%%%%%%%%%%%%%%%%%%%%%%%%%%%%%%%%%%%%%%%%%%%%%%%%%%%%%%%

\vspace{-25pt}%

\begin{tabular}[t]{lp{1in}l}
	\multirow{5}{*}{\includegraphics[width = 0.9in]{UASLP.png}} \\
																								&& UASLP \\
																								&& \\

																								&& Facultad de Agronom\'ia y Agricultura. \\
																								&& \\
																								&& \\

																								&& Examen 2 de Estad\'istica. \\
																								&& \\
																								&& Programaci\'on en R, \\ 
																								&& Regresi\'on Lineal. \\
																								&& \\
																								&& \\
																								&& \textbf{Nombre del Estudiante:} \\
\typeout{************RUBEN IS HERE DEBBUGING THE LOG}

\end {tabular}
\newpage
\noindent 


%%%%%%%%%%%%%%%%%%%%%%%%%%%%%%%%%%%%%%%%%%%%%%%%%%%%%%%%%%%%%%%%%%%%%%%%%%%%%%%%%%%%%%%%%%%%%%%%%%%%%%%%%%%%%

\section*{\textbf{\textit{A. Saca fotos a lo que escribas en papel y guarda tu programa R en tu computadora.
B. Enviar\'as a mi mail de la UASLP las fotos y el programa R (punto extra si envias el knit completo, punto 
extra si agregas gr\'aficas). 
C. El programa R vale 30\% y calificar\'e adem\'as explicaciones y notas que me ayuden a entenderlo.}}}

\section{COMPARACION DE DOS MODELOS (Setenta porciento)}

Ejemplo de datos colectados en diferente tiempo pero que tienen variables comunes, el siguiente ejercicio te 
guiar\'a en poner los datos juntos y compara los modelos.

Los datos son records mundiales de atletismo para hombres y mujeres y compararemos sus $\beta$ y
veremos si estad\'isticamente los hombres son m\'as r\'apidos que las mujeres (m\'as discusi\'on de esto en 
el seminario).

Usa los datos \textit{\textbf{en el programa adjunto en tu inbox}} y simgue las indicaciones del programa que contiene
como crear datos y regresiones:

\begin{enumerate}	
\item Lee las dos bases de datos

\item Para cada base de datos, estima la ordenada al origen y la beta que corresponden al modelo
$Y=\beta_{0}+ \beta_{1} \times X+ \varepsilon$, con Y=log(Tiempo), X=log(Distancia)

\item Pon dummy variable para sexo, y concatena los datos

\item Ejecuta el siguiente modelo de regresi\'on para la base de datos concatenada:

$log(Tiempo)=\beta_{0}+ \beta_{1} \times log(Distancia)+ \beta_{2} \times Sexo+ \beta_{3} \times Sexo*log(
Distancia)+ \varepsilon$

\end{enumerate}	

Salva tu programa y haz lo siguiente:

\begin{itemize}
\item \textbf{\textit{Reporta por escrito en papel}} el modelo estimado de la base de datos hombres y el 
modelo estimado de la base de datos mujeres

\item \textbf{\textit{Reporta por escrito en papel}} cual de los dos modelos indica m\'as rapidez

\item \textbf{\textit{Reporta por escrito en papel}} las hip\'otesis nula $H_{0}$ y alternativa $H_{A}$ de 
la intercept y slope(s) de cada uno de los dos modelos y tu decisi\'on de aceptar o no las hip\'otesis nula con un nivel de 
significancia de $\alpha$= 0.05 (usa los p-value). Reporta tambi\'en si las relaciones son crecientes o no 
con base en cada pendiente

\item \textbf{\textit{Reporta por escrito en papel}} el modelo con los datos concatenados, pero
solamente los par\'ametros (ya sea intercept, $\beta$ o ambos) significativos al nivel $\alpha$= 0.05

\item \textbf{\textit{Reporta por escrito en papel}} si la interacci\'on Sexo*log(Distancia) es significativa.
(Nota: esto se interpreta como los efectos de sexo con log(Distancia) no son significativos y podemos 
suponer que la rapidez es pararlela entre hombres y mujeres y ANCOVA es un buen modelo)

\item \textbf{\textit{Reporta por escrito en papel}} tu prediccion del tiempo que los hombres correr\'ian la Distancia de 60
y el tiempo que las mujeres correr\'ian la Distancia 200. Ten cuidado, nota que tu modelo est\'a en la escala log,
as\'i que calcular\'as con calculadora el log(Distancia de la pregunta) y luego tendr\'as que usar exponencial
porque log es base e

\item \textbf{\textit{Reporta por escrito en papel}} la \textit{Multiple R-squared}, esta la encuentras al 
ejecutar \textit{anova} (o en la tabla de resultados en SAS). Este n\'umero es deseable que este cerca de 1.00 porque en porcentaje explica 
que tan bien la variaci\'on de Y est\'a explicada por el modelo
\end{itemize}

\textit{\textbf{FIN DE EXAMEN}} \\
\textit{\textbf{FIN DE EXAMEN}} \\
\textit{\textbf{FIN DE EXAMEN}} \\


\section{Este examen fue escrito en \LaTeX}

Si quieres aprender $\LaTeX$ instala lo siguiente en este orden:

\begin{enumerate}
	\item \url{https://miktex.org/download}
	\item \url{http://www.texniccenter.org/download/}
\end{enumerate}

Guarda el \textit{tex} y \textit{png} adjuntos en tu computadora, inicia \textit{TeXnicCenter}, abre el 
\textit{tex} y haz click en el icono \textit{build and view current file} para ejecutar el \textit{tex} 
file, el resultado se abrir\'a como un \textit{pdf, html, o rtf}.
Si no aparece ning\'un pdf, html, o rtf, b\'uscalo en el folder donde est\'a tu tex file.

Asegurate que tienes pdf instalado, o alguna version gratis como pdf reader, o pdf Sumatra.

Experimenta con el \textit{tex} file y pronto podr\'as escribir tus propios documentos en \LaTeX.

%%%%%%%%%%%%%%%%%%%%%%%%%%%%%%%%%%%%%%%%%%%%%%%%%%%%%%%%%%%%%%%%%%%%%%%%%%%%%%%%%%%%%%%%%%%%%%%%%%%%%%%%%%%%%%
\end{document} 
%%%%%%%%%%%%%%%%%%%%%%%%%%%%%%%%%%%%%%%%%%%%%%%%%%%%%%%%%%%%%%%%%%%%%%%%%%%%%%%%%%%%%%%%%%%%%%%%%%%%%%%%%%%%%%
