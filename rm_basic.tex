%%%%%%%%%%%%%%%%%%%%%%%%%%%%%%%%%%%%%%%%%%%%%%%%%%%%%%%%%%%%%%%%%%%%%%%%%%%%%%%%%%%%%%%%%%%%%%%%%%%%%%%%%%%%%%%%%%%%%%%%%%%%%%%%%%%
%https://nw360.blogspot.com/2017/03/set-up-pdf-viewer-in-texniccenter.html
%%%%%%%%%%%%%%%%%%%%%%%%%%%%%%%%%%%%%%%%%%%%%%%%%%%%%%%%%%%%%%%%%%%%%%%%%%%%%%%%%%%%%%%%%%%%%%%%%%%%%%%%%%%%%%%%%%%%%%%%%%%%%%%%%%%
\documentclass[letterpaper,11pt]{article}\usepackage[]{graphicx}\usepackage[]{color}
%% maxwidth is the original width if it is less than linewidth
%% otherwise use linewidth (to make sure the graphics do not exceed the margin)
\makeatletter
\def\maxwidth{ %
  \ifdim\Gin@nat@width>\linewidth
    \linewidth
  \else
    \Gin@nat@width
  \fi
}
\makeatother

\definecolor{fgcolor}{rgb}{0.345, 0.345, 0.345}
\newcommand{\hlnum}[1]{\textcolor[rgb]{0.686,0.059,0.569}{#1}}%
\newcommand{\hlstr}[1]{\textcolor[rgb]{0.192,0.494,0.8}{#1}}%
\newcommand{\hlcom}[1]{\textcolor[rgb]{0.678,0.584,0.686}{\textit{#1}}}%
\newcommand{\hlopt}[1]{\textcolor[rgb]{0,0,0}{#1}}%
\newcommand{\hlstd}[1]{\textcolor[rgb]{0.345,0.345,0.345}{#1}}%
\newcommand{\hlkwa}[1]{\textcolor[rgb]{0.161,0.373,0.58}{\textbf{#1}}}%
\newcommand{\hlkwb}[1]{\textcolor[rgb]{0.69,0.353,0.396}{#1}}%
\newcommand{\hlkwc}[1]{\textcolor[rgb]{0.333,0.667,0.333}{#1}}%
\newcommand{\hlkwd}[1]{\textcolor[rgb]{0.737,0.353,0.396}{\textbf{#1}}}%
\let\hlipl\hlkwb

\usepackage{framed}
\makeatletter
\newenvironment{kframe}{%
 \def\at@end@of@kframe{}%
 \ifinner\ifhmode%
  \def\at@end@of@kframe{\end{minipage}}%
  \begin{minipage}{\columnwidth}%
 \fi\fi%
 \def\FrameCommand##1{\hskip\@totalleftmargin \hskip-\fboxsep
 \colorbox{shadecolor}{##1}\hskip-\fboxsep
     % There is no \\@totalrightmargin, so:
     \hskip-\linewidth \hskip-\@totalleftmargin \hskip\columnwidth}%
 \MakeFramed {\advance\hsize-\width
   \@totalleftmargin\z@ \linewidth\hsize
   \@setminipage}}%
 {\par\unskip\endMakeFramed%
 \at@end@of@kframe}
\makeatother

\definecolor{shadecolor}{rgb}{.97, .97, .97}
\definecolor{messagecolor}{rgb}{0, 0, 0}
\definecolor{warningcolor}{rgb}{1, 0, 1}
\definecolor{errorcolor}{rgb}{1, 0, 0}
\newenvironment{knitrout}{}{} % an empty environment to be redefined in TeX

\usepackage{alltt}
\usepackage{float}
\usepackage{graphicx}
\usepackage{amsmath}
\usepackage{amssymb}
\usepackage{bm}         
\usepackage{amsthm}
\usepackage{afterpage}
\usepackage[table, usenames, dvipsnames]{xcolor}
\usepackage{latexsym}

%\usepackage[framemethod=0,ntheorem]{mdframed}

\usepackage{etexcmds}
\usepackage{fullpage}
\usepackage{setspace}
\usepackage{multirow}
\usepackage{rotating}
\usepackage{pdflscape}
\usepackage{fancyhdr}
\usepackage[utf8]{inputenc}
\usepackage{lscape}
\usepackage{color}
\usepackage{tabularx}
\usepackage{array}
\usepackage{longtable}
\usepackage[english]{babel}
\usepackage{csquotes}

\usepackage[font=small,labelfont=bf,skip=0pt]{caption}
%\captionsetup[table]{skip=0.2pt}
%\captionsetup[table]{aboveskip=0pt}
%\captionsetup[table]{belowskip=-15pt}

\usepackage{subcaption}

%\captionsetup[subtable]{skip=2pt}

\usepackage[tableposition=top]{caption}
\usepackage[toc,page]{appendix}
\usepackage{amsfonts}

\setcounter{MaxMatrixCols}{30}
\providecommand{\U}[1]{\protect\rule{.1in}{.1in}}

\newcommand{\tablefont}{\fontsize{3mm}{3mm}\selectfont}
\MakeOuterQuote{"}

\newcommand{\MC}{\multicolumn}
\newcommand{\MR}{\multirow}

\headheight 15pt
\headsep 2em

\renewcommand{\footrulewidth}{0pt}
\newtheorem{remark}{Remark}
\newtheorem{comment}{\textbf{Comment}}
\newtheorem{definition}{Definition}
\newtheorem{discussion}{Discussion:}
\newtheorem{claim}{Claim}
\newtheorem{question}{Question}
\newtheorem{answer}{Answer}
\newcolumntype{T}{>{\tiny}l}
\newcolumntype{M}{>{\centering\arraybackslash}m{3.1cm}}

\usepackage{fancyhdr}
\fancyhf{}
\pagestyle{fancy} %added by CDE to allow headers, footers to work
%\rhead{Share\LaTeX}
%\lhead{Guides and tutorials}


\usepackage[pdftex]{hyperref}   
\hypersetup{colorlinks, citecolor=Violet, linkcolor=Mahogany, urlcolor=blue}

\rfoot{Page \thepage}
\lfoot{DRAFT-DELIBERATIVE-CONFIDENTIAL}  

%%%%%%%%%%%%%%%%%%%%%%%%%%%%%%%%%%%%%%%%%%%%%%%%%%%%%%%%%%%%%%%%%%%%%%%%%%%%%%%%%%%%%%%%%%%%%%%%%%%%%%%%%%%%%%%%%%%%%%%%%%%%%%%%%%%
\IfFileExists{upquote.sty}{\usepackage{upquote}}{}
\begin{document}
%%%%%%%%%%%%%%%%%%%%%%%%%%%%%%%%%%%%%%%%%%%%%%%%%%%%%%%%%%%%%%%%%%%%%%%%%%%%%%%%%%%%%%%%%%%%%%%%%%%%%%%%%%%%%%%%%%%%%%%%%%%%%%%%%%%


\vspace{-25pt}%

\begin{tabular}[t]{lp{1in}l}
	\multirow{5}{*}{\includegraphics[width = 2in]{logo.png}} && Department \leavevmode  \\
																								&& Dep \\
																								&& Center\\
\end{tabular}

\typeout{************RUBEN IS HERE DEBBUGING THE LOG}

\leavevmode \newline \vspace{15pt} \newline {\Large \textsc{Generalized Estimating Equations and Linear Mixed Models}}\newline\vspace{0.015in}

\begin{tabular}[h!]{p{2in} p{10in}}
	\rule{0pt}{4ex}\textbf{Reporting to:}          & name1  \\
																							   & position \\
                                                 & \\
	\rule{0pt}{4ex}\textbf{Key Words:}  					 & Individual and population approaches\\
																								 & Sandwich estimator, REML, Quasi Likelihood \\
 \mbox{$\quad$} \\
 \mbox{$\quad$} \\
\end{tabular}

\newpage
\noindent 

%%%%%%%%%%%%%%%%%%%%%%%%%%%%%%%%%%%%%%%%%%%%%%%%%%%%%%%%%%%%%%%%%%%%%%%%%%%%%%%%%%%%%%%%%%%%%%%%%%%%%%%%%%%%%%%%%%%%%%%%%%%%%%%%%%%

%2018-10-23

\section*{EXAMEN DE PRACTICA}

Ejercicios de practica de examen

\section{USO DE DATOS EN R}

\textbf{\textit{Ejercicios 1}}:\\

Para cada conjunto de los siguientes datos ajusta un modelo de recta de \textbf{\textit{minimos cuadrados}} tambien conocidos como 
\textbf{\textit{Ordinary Least Squares (OLS)}}, esto es, lee los datos en R y ejecuta \textbf{\textit{linear models (lm)}} para obtener la \textbf{\textit{ordenada al origen (intercept)}} y \textbf{\textit{pendiente (slope)}}. Una vez que tengas estos dos numeros, escribe el modelo a mano usando la notacion del modelo estimado:

\begin{equation}
\hat{Y}= \hat{\beta_{0}} + \hat{\beta_{1}} X
\end{equation}
	
	
donde $\hat{\beta_{0}}$ la reemplazas por intercept y $\hat{\beta_{1}}$ por slope.

\section{USO DE FORMULAS}

Estas son las formulas para obtener intercept y slope con calculadora y se llaman \textbf{\textit{Ecuaciones Normales}}:

\begin{itemization}
\item
	\begin{equation}
	\sum{Y_{i}}= n(\hat{\beta_{0}}) + (\sum{X_{i}) \hat{\beta_{1}}
	\end{equation}

\item
	\begin{equation}
	\sum{X_{i} Y_{i}}= (\sum{X_{i}) \hat{\beta_{0}} + (\sum{X_{i}^2) \hat{\beta_{1}}
	\end{equation}
	
\end{itemization}

Las soluciones son las siguientes dos ecuaciones:
	
	\begin{equation}
	\hat{\beta_{1}}= \frac {\sum (X_{i} - \overline{X})* (Y_{i} - \overline{Y})} {(\sum X_{i} -\overline{X})^2}
	\end{equation}

	\begin{equation}
	\hat{\beta_{0}}= \overline{Y} - \hat{\beta_{1}} \overline{X}
	\end{equation}
	
Con $\overline{Y}, \overline{Y}$ los promedios de las observaciones $Y_{i}, X_{i}$	respectivamente.\\
\textbf{\textit{Ejercicio 2}}:\\
	
Si Y tiene valores 1, 4, 6 y X tiene valores 1, 3, 5, usa las formulas de arriba y calcula $\hat{\beta_{0}}$ y $\hat{\beta_{1}}$ \\
En R como calculadora es asi:

X=c(1,3,5) \\
Y=c(1,4,6) \\
rr=data.frame(y,x) \\
rr \\
plot(Y~X, data=rr) \\
summary(lm(Y~X, data=rr)) \\ \\

sumXY=(1*1) + (3*4)+ (5*6) \\
sumXsumY=(1+3+5)* (1+4+6) \\
sumXX=1+ 9+ 25 \\
sumX=1+3+5 \\
sumY=1+4+6 \\
n=3 \\ \\

beta1HAT= ( sumXY - ( sumXsumY / n) ) / (sumXX- ${sumX}^2$) /n) \\
beta1HAT \\
beta0HAT=(sumY/n) - beta1* sumX/n \\
beta0HAT \\ \\

Notas:
El modelo teorico es $Y= \beta_{0} + \beta_{1} \times X + \epsilon$ \\Ahi aceptamos que hay un error al que llamamos $\epsilon$
Una vez que tenemos datos y hacemos los calculos obtenemos una aproximacion al modelo teorico llamado 
modelo estimado: $\hat{Y}= \hat{\beta_{0}} + \hat{\beta_{1}} \times X$ \\ 
los parametros $\hat{\beta_{0}} + \hat{\beta_{1}}$ se reemplazan ahora por la ordenada al origen y pendiente obtenida por
calculos ya sea a mano o con R. El modelo estimado ya no tiene escrito el error $\epsilon$,
en vez, el error esta ahora estimado (calculado) como la suma de residuos:  

	\begin{equation}
	SS(Res)= \sum(Y_{i}- \hat{Y_{i}})^2
	\end{equation}
	
	donde i=1, 2, 3,... son las observaciones o renglones en los datos observados: $Y_{i}$ son las observaciones Y en nuestros datos,
	y $\hat{Y_{i}}$ son los valores estimados de Y.
	
	
\section{MODELO PROBABILISTICO O ESTADISTICO}
Las dos secciones anteriores tratan de la geometria de ajustar datos a un modelo teorico 

	\begin{equation}
$Y= \beta_{0} + \beta_{1} \times X + \epsilon$
	\end{equation}

Agregaremos ahora probabilidad a este modelo suponiendo lo siguiente: \\

\begin{itemization}
\item Los errores de cada observacion i=1, 2, 3... tienen una distribucion probabilistica \textbf{\textit{Normal}} con valor esperado cero y varianza $\sigma^{2}$, esto se escribe como

	\begin{equation}
\epsilon_{i} \sim \mathcal{N}(0,\sigma^{2})
	\end{equation}

\item Los errores son independientes mutuamente, esto es, la probabilidad de $\epsilon_{i}$ no es afectada por la probabilidad de 
$\epsilon_{j}$ con i, j observaciones 1, 2, 3... y j $\neq$ i.

Como la variable X no es probabilistica, la variable Y hereda estas propiedades de los errores $\epsilon$:

	\begin{equation}
Y_{i} \sim \mathcal{N}( (\beta_{0} + \beta_{1} X),\sigma^{2})
	\end{equation}
	
Con estas suposiciones, nuestras Ecuaciones Normales nos daran los \textbf{\textit{Best Linear Unbiased Estimators (BLUE)}},
o sea, cualquier otra manera de estimar linealmente y sin sezgo la ordenada al origen y la pendiente nos daria estimadores con una mayor varianza. Entonces, nuestros estimadores son muy buenos geometricamente por tener la distancia minima a la recta estimada, y son muy buenos porque ahora tienen esta propiedad probabilistica deseable de ser BLUE.\\

\textbf{\textit{Ejercicios 3:}}\\

De los datos 1, ejecuta tu modelo, por ejemplo para los datos cars: rm.lm $<-$ lm(dist~speed, data=cars).
Ahora ejecuta summary(rm.lm)

Esta tabla se llama \textbf{\textit{Analisis de Varianza (ANOVA)}} y tiene aparte de los coeficientes ordenada al origen y slope,
Std. Error, t value y Pr( $> \abs{t}$).

Observa los valores de Pr( $> \abs{t}$) que corresponden a la ordenada al origen y a la slope.
Preguntate si alguno de estos numeros es menor que 0.05.

Observa tambien el valor Adjusted $R-squared$ y pregunate si este valor esta cercano a 1.
Observa tambien el valor $F-statistic$ y el $p-value$, y preguntate si este valos es menor que 0.05

\section{X COMO VARIABLE ALEATORIA, COMO VARIABLE NO ALEATORIA}
En esta seccion pondremos mas atencion en la formulacion del modelo teorico

R tiene unos datos llamados women que es promedio de estaturas (height) y pesos (weight) de mujeres de los EU,
tiene solo 15 observaciones.

\textbf{\textit{Ejercicios 4}}:\\

Ejecuta plot(height, weight, data=women)

Haz un modelo de regresion, tu escoge cual es tu Y y cual es tu X y reporta tu tabla ANOVA con tu modelo estimado como en los ejercicios 
USOS DE DATOS EN R

Ahora, revisa si estos datos son independientes, esto es, si la probabilidad de Y depende de X ejecutando una 
prueba de $\CHI_^{2}$, recuerda que la hipotesis nula es $H_{0}$: P(Y|X)= P(Y) de que Y y X son independientes, 
y la $H_{A}$: de que  P(Y|X) $\neq$ P(Y). Usaremos un nivel de significancia de $\alpha$= 0.05.

Reporta los valores $\CHI_^{2}$, grados de libertad y p-value. Escribe cual es tu conclusion, esto es, aceptas $H_{0}$ o no

Ahora haz una regresion, aunque en el ejercicio anterior usamos de que height y weight son variables aleatorias o probabilisticas, en nuestro modelo de regresion supondremos que la Y es aleatoria y la X no lo es.

Reporta lo que observas en la tabla de ANOVA y explica cuales son las hipotesis $H_{0}$, y que es lo que te dice F-statistic junto su p-value.
Lo que reportes debe de ser que la ordenada al origen y la pendiente son diferentes de cero significativamente (al nivel de
significancia $\alpha$= 0.05, y que el modelo es significativamente diferente de cero, esto es:

En el full model ($Y= \beta_{0} + \beta_{1} \times X + \epsilon$)

$H_{0}: \beta_{0}$=0 y $\beta_{1}$=0
$H_{A}$: la negacion de H0.

\section{DESIGN MATRIXY ALGEBRA LINEAL}
Los datos women son:

  height weight
     58    115
     59    117
     60    120
     61    123
     62    126
     63    129
     64    132
     65    135
     66    139
     67    142
     68    146
     69    150
     70    154
     71    159
     72    164

La observacion 1 es $Y_{1}=58, X_{1}=115, \ldots$ la observacion 15 es $Y_{15}=72, X_{15}= 164$

Y se puede ver como el vector Y es:

	\[\pmb{Y}=
	\begin{bmatrix}
	58  \\
	59  \\
	\vdots \\
	72 	\\
	\end{bmatrix}
	\] 


y para nuestro modelo con $\beta_{0} y \beta_{1}$ usamos la design matrix siguiente:


	\[\pmb{X}=
	\begin{bmatrix}
	1 			& 115 	\\
	1 			& 117 	\\
	\vdots 	& \vdots  \\
	1 			& 164 	\\
	\end{bmatrix}
	\]
\\ \\

El modelo de regresion lineal se escribe con matrices de la siguiente manera:

$Y= \beta_{0} + \beta_{1} \times X + \epsilon$

donde $\epsilon$ es tambien un vector.\\

\left[ 
\begin{array}{c} 
Y_{1} \\ 
Y_{2} \\
\vdots \\
Y_{n} \\ 
\end{array} 
\right] 
= 
\begin{bmatrix} 
1 & X_{1} \\ 
1 & X_{2} \\
\vdots & \vdots \\
1 & X_{n} \\ 
\end{bmatrix}
 
\times 

\left[ 
\begin{array}{c} 
\beta_{1} \\ 
\beta_{2} \\
\end{array} 
\right]
+
\left[
\begin{array}{c} 
\epsilon_{1} \\ 
\epsilon_{2}  \\
\vdots \\
\epsilon_{n}
\end{array}
\right]

\\ \\

Por lo tanto, el modelo se puede escribir coordenada por coordenada:

$Y_{i}= \beta_{0i}+ \beta_{1i} \times X_{i} + \epsilon_{i}$

Cuando tenemos los datos, el modelo ajustado es:\\

$\hat{Y_{i}}= \hat{\beta_{0}} + \hat{\beta_{1}} \times X_{i}}$ con i=1, 2, 3,..., 15.\\

\textbf{\textit{Ejercicios 5}}:\\

Usando tu modelo estimado $\hat{Y}= \hat{\beta_{0}} + \hat{\beta_{1}} \times X$ \\ 
sutituye en tu modelo los dos primeros valores de X y escibe cuales son los valores esperados de Y.

%%%%%%%%%%%%%%%%%%%%%%%%%%%%%%%%%%%%%%%%%%%%%%%%%%%%%%%%%%%%%%%%%%%%%%%%%%%%%%%%%%%%%%%%%%%%%%%%%%%%%%%%%%%%%%%%%%%%%%%%%%%%%%%%%%%

\end{document} 
%%%%%%%%%%%%%%%%%%%%%%%%%%%%%%%%%%%%%%%%%%%%%%%%%%%%%%%%%%%%%%%%%%%%%%%%%%%%%%%%%%%%%%%%%%%%%%%%%%%%%%%%%%%%%%%%%%%%%%%%%%%%%%%%%%%
