%%%%%%%%%%%%%%%%%%%%%%%%%%%%%%%%%%%%%%%%%%%%%%%%%%%%%%%%%%%%%%%%%%%%%%%%%%%%%%%%%%%%%%%%%%%%%%%%%%%%%%%%%%%%%%%%%%%%%%%%%%%%%%%%%%%
\documentclass[letterpaper,11pt]{article}

\usepackage{float}
\usepackage{graphicx}
\usepackage{amsmath}
\usepackage{amssymb}
\usepackage{bm}         
\usepackage{amsthm}
\usepackage{afterpage}
\usepackage[table, usenames, dvipsnames]{xcolor} %added by CDE
\usepackage{latexsym}
\usepackage[framemethod=0,ntheorem]{mdframed}
\usepackage{multirow}
\usepackage{fullpage}
\usepackage{setspace}
\usepackage{multirow}
\usepackage{rotating}
\usepackage{pdflscape}
\usepackage{fancyhdr}
\usepackage[utf8]{inputenc}
\usepackage{lscape}
\usepackage{color}
\usepackage{tabularx}
\usepackage{array}
\usepackage{longtable}
\usepackage[english]{babel}
\usepackage{csquotes}
\usepackage[font=small,labelfont=bf,skip=0pt]{caption}
\captionsetup[table]{skip=0.2pt}
\captionsetup[table]{aboveskip=0pt}
\captionsetup[table]{belowskip=-15pt}

\usepackage{subcaption}

\captionsetup[subtable]{skip=2pt}

\usepackage[tableposition=top]{caption}
\usepackage[toc,page]{appendix}
\usepackage{amsfonts}

\setcounter{MaxMatrixCols}{30}
\providecommand{\U}[1]{\protect\rule{.1in}{.1in}}

\newcommand{\tablefont}{\fontsize{3mm}{3mm}\selectfont}
\MakeOuterQuote{"}

\newcommand{\MC}{\multicolumn}
\newcommand{\MR}{\multirow}

\headheight 15pt
\headsep 2em

\renewcommand{\footrulewidth}{0pt}
\newtheorem{remark}{Remark}
\newtheorem{comment}{\textbf{Comment}}
\newtheorem{definition}{Definition}
\newtheorem{discussion}{Discussion:}
\newtheorem{claim}{Claim}
\newtheorem{question}{Question}
\newtheorem{answer}{Answer}
\newcolumntype{T}{>{\tiny}l}
\newcolumntype{M}{>{\centering\arraybackslash}m{3.1cm}}

\usepackage{fancyhdr}
\fancyhf{}
\pagestyle{fancy} %added by CDE to allow headers, footers to work
%\rhead{Share\LaTeX}
%\lhead{Guides and tutorials}


\usepackage[pdftex]{hyperref}   
\hypersetup{colorlinks, citecolor=Violet, linkcolor=Mahogany, urlcolor=blue}

\rfoot{Page \thepage}
\lfoot{DRAFT-DELIBERATIVE-CONFIDENTIAL}  

%%%%%%%%%%%%%%%%%%%%%%%%%%%%%%%%%%%%%%%%%%%%%%%%%%%%%%%%%%%%%%%%%%%%%%%%%%%%%%%%%%%%%%%%%%%%%%%%%%%%%%%%%%%%%%%%%%%%%%%%%%%%%%%%%%%
\begin{document}
%%%%%%%%%%%%%%%%%%%%%%%%%%%%%%%%%%%%%%%%%%%%%%%%%%%%%%%%%%%%%%%%%%%%%%%%%%%%%%%%%%%%%%%%%%%%%%%%%%%%%%%%%%%%%%%%%%%%%%%%%%%%%%%%%%%

\vspace{-25pt}%
\begin{tabular}[t]{lp{1in}l}
	\multirow{5}{*}{\includegraphics[width = 2in]{fdalogo.png}} &  & U.S. Department of Health and Human Services\\
																								&  & Food and Drug Administration \\
																								&  & Center for Tobacco Products \\
																								&  & Office of Sciences \\
																								&  & Division of Population Health Science \\
																								&  & Statistics Branch \\
\end{tabular}

\newline\vspace{15pt} \newline{\Large \textsc{Applied Regression Analysis}}\newline\vspace{0.015in}

\begin{tabular}[h!]{p{2in} p{10in}}
	\rule{0pt}{4ex}\textbf{Reporting to:}          & Antonio Paredes, MA, MS  \\
																							   & Lead Mathematical Statistician \\
																							   & Division of Population Health Science \\
                                                 & Statistics Branch \\
																							   & \\
	\rule{0pt}{4ex}\textbf{Subject:}               & Applied Regression \\
	\rule{0pt}{4ex}\textbf{Date:}                  & \today \\
	\rule{0pt}{4ex}\textbf{Statistical Reviewer:}  & Ruben Montes de Oca, MS \\
																							   & Mathematical Statistician \\
																							   & Division of Population Health Science \\
                                                 & Statistics Branch \\
																							   & \\
	\rule{0pt}{4ex}\textbf{Branch Chief:}          & George Rochester, PhD \\
																							   & Chief, Statistics Branch \\
																							   & Division of Population Health Science \\
                                                 & Statistics Branch \\
                                                 & \\
	\rule{0pt}{4ex}\textbf{Key Words:}  					 & Regression \\
 \mbox{$\quad$} \\
 \mbox{$\quad$} \\
\end{tabular}

\newpage
\noindent 

%%%%%%%%%%%%%%%%%%%%%%%%%%%%%%%%%%%%%%%%%%%%%%%%%%%%%%%%%%%%%%%%%%%%%%%%%%%%%%%%%%%%%%%%%%%%%%%%%%%%%%%%%%%%%%%%%%%%%%%%%%%%%%%%%%%


%page 2: %%%%%%%%%%%%%%%%%%%%%%%%%%%%%%%%%%%%%%%%%%%%%%%%%%%%%%%%%%%%%%%%%%%%%%%%%%%%%%%%%%%%%%%%%%%%%%%%%%%%%%%%%%%%%%%%%%%%%%%%%%%%%%%%%%%
\section*{CHAPTER 3: MULTIPLE REGRESSIONS WITH MATRICE}

\section{THE MODEL}

We will work on this linear model with p independent variables X_{i1}, X_{i2}, X_{ip}:

	\begin{equation} 
	\begin{gathered}
	$Y_{i}= \beta_{0} + \beta_{1} X_{i1}  + \beta_{2} X_{i2} + \ldots + \beta_{p} X_{ip}  + \epsilon_{i}$ 
	\end{gathered}
	\end{equation}

\noindent 
In this model we have observation units $Y_{i}$ with i=1, 2, $\ldots$, n, p independent variables X_{i1}, X_{i2}, X_{ip} and p'=p+1 
parameters $\beta_{0}, \beta_{1},..., \beta_{p}$. We assume that n $>$ p'.
We can express this model using matrices as follows:
	
	\begin{equation} 
	\begin{gathered}
	\pmb{Y= X \beta + \epsilon} 
	\end{gathered}
	\end{equation}	
	
where:
\\ \\
	
	\pmb{Y}: the n $\times$ 1 column vector of observations $Y_{i}$:
	
	\[\pmb{Y}=
	\begin{bmatrix}
	Y_{1}		\\
	Y_{2}		\\
	\vdots 	\\
	Y_{n}		\\
	\end{bmatrix}
	\]
\\ \\
	
	\pmb{X}: the n $\times$ p' matrix of a column of ones and p columns of the independent variables
\\ \\

	\[\pmb{X}=
	\begin{bmatrix}
	1 			& X_{11} & X_{12} & \ldots & X_{1p}	\\
	1 			& X_{21} & X_{22} & \ldots & X_{2p}	\\
	\vdots 	& \vdots & \vdots & \vdots & \vdots \\
	1 			& X_{n1} & X_{n2} & \ldots & X_{np}	\\
	\end{bmatrix}
	\]
\\ \\
	
	\pmb{$\beta$}: the p' $\times$ 1 vectors of parameters to be estimated
	\\ \\
	\[\pmb{\beta}=
	\begin{bmatrix}
	\beta_{0} \\
	\beta_{1} \\
	\vdots 	\\
	\beta_{n} \\
	\end{bmatrix}
	\]
\\ \\	
	
	\pmb{$\epsilon$}: the n $\times$ 1 vectors of parameters to be estimated

	\[\pmb{\epsilon}=
	\begin{bmatrix}
	\epsilon_{1} \\
	\epsilon_{2} \\
	\vdots 	\\
	\epsilon_{n} \\
	\end{bmatrix}
	\]
We assume that the matrix X is a full rank matrix, that is, the columns of X are linearly independent and the model is called full rank.
The non-full rank models will be covered on Chapter 9.
Each column of X contains the values for a particular independent variable and multiplying the first row of X by $\beta$, and adding the first element of $\epsilon$ confirms that the model for the first observation is:

	\begin{equation} 
	\begin{gathered}
	$Y_{i}= \beta_{0} + \beta_{1} X_{11}  + \beta_{2} X_{12} + \ldots + \beta_{p} X_{1p}  + \epsilon_{1}$ 
	\end{gathered}
	\end{equation}

It is common to assume that $\epsilon_{i}$ are independent and identically distributed as normal random variables with mean zero and variance $\sigma^2$. Since $\epsilon_{i}$ are assumed to be independent of each other, the covariance between $\epsilon_{i}$ and $\epsilon_{j}$ is zero for any i $\neq$ j. The joint probability density function of $\epsilon_{1}, \epsilon_{2}, . . . , \epsilon_{n}$ is

	\begin{gathered}
	\begin{equation}
	\prod_{i=1}^{n} [(2 \pi) ^{-1/2} \sigma ^{-1} exp(-\epsilon_{i} ^2/2 {\sigma} ^2)] = (2 \pi) ^{-n/2} \sigma ^{-n} exp(- \sum_{i=1} ^{n} 	
	\epsilon_{i}^2/2 \sigma^2)
	\end{gathered}
	\end{equation}
	
Following our assumptions of \epsilon and that the X and \beta are fixed constants:
	
	\begin{itemize}
	\item Yi is a normal random variable with mean $\beta_{0}+\beta_{1} X_{i1}+\beta_{2} X_{i2}+\ldots+ \beta_{p} X_{ip}$ and variance $\sigma 
	^{2}$
	\item The $Y_{i}, Y_{j}$ are independent of each other for any i $\neq$ j
	\end{itemize}
	
\section{NORMAL EQUATIONS, $\hat{Y}$ AND RESIDUALS VECTORS}

In matrix notation, the normal equations are written as:
\\ \\

X' X $\hat{\beta}}$ = X' Y
\\ \\

The normal equations are always consistent and hence will always have a solution of the form
$\hat{\beta} = (X' X)^{-} X' Y$, where $A^{-}=(X' X)^{-}$ is a generalized inverse matrix, that is, a matrix that satisfies A $A^{-}$ A= A
If X' X has an inverse, then the normal equations have a unique solution given by $\hat{\beta} = (X' X) ^{-1} (X' Y)$

\\ \\
The vector of estimated means of the dependent variable Y for the values $\hat{Y}$ is computed as
$\hat{Y} = X \hat{\beta}$.
The residuals vector e reflects the lack of agreement between the observed Y and the estimated $\hat{Y}$:
e = Y - $\hat{Y}$.

\section{PROPERTIES OF LINEAR FUNCTIONS OF RANDOM VECTORS}

Note that $\hat{\beta}, \hat{Y}$ and e are random vectors because they are functions of the random vector Y.

If Z=$(z_{1} \ldots z_{n})'$ is a random vector $n \times 1$ the mean of Z is an n $\times$ 1 vector with i-th coordinate 
$E(z_{i})$.
The variance-covariance matrix $V_{Z}$=Var(Z) for Z is defined as an n $\times$ n symmetric matrix with the diagonal
elements equal to the variances of the random variables and the (i, j)th off-diagonal element equal to the covariance 
between $z_{i}$ and $z_{j}$:
\\ \\

	\[V_{Z}=
	\begin{bmatrix}
	Cov(z_{1}, z_{1}) & Cov(z_{1}, z_{2}) & \ldots Cov(z_{1}, z_{n-1}) & Cov(z_{1}, z_{n})\\
	Cov(z_{2}, z_{1}) & Cov(z_{2}, z_{2}) & \ldots Cov(z_{2}, z_{n-1}) & Cov(z_{2}, z_{n})\\
	\vdots 						& \vdots 						& \ldots \vdots 						 & \vdots						\\
	Cov(z_{n}, z_{1}) & Cov(z_{n}, z_{2}) & \ldots Cov(z_{n}, z_{n-1}) & Cov(z_{n}, z_{n})\\
	\end{bmatrix}
	\]
\\ \\

We will develop the multivariate normal distribution and present some properties of multivariate normal random vectors.
Suppose $z_{1}, z_{2}, \ldots z_{n}$ are independent normal random variables with mean zero and variance $\sigma^{2}$
Then, the random vector Z=($z{1}, z_{2}, \ldots z_{n}$) has a multivariate normal distribution with mean 0=(0, \ldots 0)'
and variance-covariance matrix $V_{z} = I \sigma^{2}$, which is denoted as Z $\sim \mathcal{N}(0,I \sigma^{2})$
The probability density function of Z is given in equation () and can also be expressed as

	\begin{gathered}
	\begin{equation}
	(2 \pi) ^{-n/2} |I \sigma ^{2}|^{-1} exp(Z' (I \sigma ^{2})^{-1} Z/2)
	\end{gathered}
	\end{equation}

It is a general result that if $\pmb{U}$ is a function $\pmb{U=A Z+ b}$ with A a k $\times$ n matrix of constants and $\pmb{b}$ a k $\times$ 1 vector of constants, then $\pmb{U} \sim \mathcal{N}(\mu_{U}, V_{U})$ and the probability density function of U is:
\\ \\

	\begin{gathered}
	\begin{equation}
	(2 \pi) ^{-k/2} |V_{U}| ^{-1/2} exp ^{(-1/2) ( [U- \mu_{U}]' V_{U} ^{-1} [U- \mu_{U}] )
	\end{gathered}
	\end{equation}


Using matrices, we can derive the expectation and variance of a sample mean. Suppose $Y_{1}, Y_{2}, \ldots , Y_{n}$ are independent random variables with mean $\mu$ and variance $\sigma^{2}$. Then, for Y = ($Y_{1}, Y_{2}, \ldots , Y_{n}$)'

E(Y)= ($\mu, \ldots , \mu$)' and Var(Y)= I $\sigma ^{2}$

\section{PROPERTIES OF REGRESSION ESTIMATES}


Recall that in $Y = X \beta + \epsilon$, we assumed that the $\epsilon$ are independent and identically distributed (iid) as 
$\sim \mathcal{N}(0,I \sigma^{2})$. Since $\beta$ is a constant, 

	\begin{gathered}
	\begin{equation}
	Y \sim \mathcal{N}( x \beta, \sigma^{2} I)
	\end{gathered}
	\end{equation}

and the joint probability density function of Y is:

	\begin{gathered}
	\begin{equation}
	(2 \pi) ^{-n/2} |I \sigma^{2}| ^{-1/2} exp ^{(-1/2) ( [Y- X \beta]' (I \sigma^{2})^-{1} (T- X \beta)
	\end{gathered}
	\end{equation}

Since $\hat{\beta}$=$[(X' X)^{-1} X'] E(Y)$, then the estimates of the regression coefficients are linear functions of the dependent variable Y.
So, if the model $Y = X \beta + \epsilon$ is correct, the expected value of Y is X $\beta$ and the expected value of $\hat{\beta}$ is:

E($\hat{\beta}$)= $[(X'X)^{-1} X] E(Y)$ = $[(X'X)^{-1} X']X \beta= [(X'X)^{-1} X'X] \beta= \beta$.

This shows that $\hat{\beta}$ is an unbiased estimator of $\beta$ for the assumed model $Y = X \beta + \epsilon$.
Continuing with this model, $Var(\hat{\beta})=(X X')^{-1} X X' (X' X)^{-1} \sigma^{2}= (X X')^{-1} \sigma^{2}$
and we arrived to $\hat{\beta} \sim \mathcal{N}(\beta, (X X')^{-1} \sigma^{2})$

Recall that the residuals are e= Y - P Y, where $P= [X (X' X)^{-1} X']$, then E(e)= (1-P) E(Y)= $(I-P) X \beta$ = $(X- PX) \beta= (X' - X) \beta$=0. Thus the residuals are a random variable with mean a zero vector $n \times 1$.
The variance-covariance matrix of the residual vector e is:
Var(e)= (I - P) $\sigma^{2}$, again, suing that (I-P) is an idempotent symmetric matrix. If the vector $\epsilon$ is normally distributed, then e $\sim \mathcal{N}(0, (I-P) \sigma^{2})$


\section{EXERCISES}

1. The dat1.xlsx has 768 observations. We will use bmi as our Y variable glucose, diastolic, triceps, insulin, diabetes, age, exposure, and sex as our X variables to fit a linear regression model model $Y_{i} = X_{i} \beta + \epsilon_{i}$, for i=1, 2, \ldots, 768 using matrix algebra. The formulas used are in figures created by R knit.
\\ \\

	\begin{itemize}
	\item Figure 1: summary of the data 

	\item Figure 2: first rows of the 768 $\times$ 9 design matrix X, and of the observed vector 768 $\times$ 1 vector Y

	\item Figure 3: estimates $\hat{\beta}$ of $\beta$ and formulas used for these estimates
	
	\item Figure 4: Variance-covariance matrix and stderr

	\item Figure 5: t-test for $\hat{\beta_{0}}$ to $\hat{\beta_{8}}$ and p-values

	\item Figure 6: Sum of Squares, Mean Squares, and F sttaistics with p-value

	\item Figure 7: R Square, Adjusted R Square, and CV

	\item Figure 8: Results using lm, compare to figures 1 to 7
	
	\item Figure 9: Excercise 3.2.A. det[(X'X)^{-1} X'] $\neq$ 0 thus the Normal Equations have a unique solution  

	\item Figure 10: Excercise 3.2.B. det[(X'X)^{-1} X'] $\neq$ 0 thus the Normal Equations have a unique solution  

	\item Figure 11: Excercise 3.2.C. $det[(X'X)^{-1} X']$ = 0 thus the Normal Equations do not have a unique solution  
	
	\item Figure 12: Excercise 3.4 Find n and $\sum{X_{i}}$ for the matrix X
	\end{itemize}
		
	\begin{figure}[h!]
	\centering
	\includegraphics[scale=1.5]{fig1}
	\caption{Summary of dat1}
	\end{figure}
	


	\begin{figure}[h!]
	\centering
	\includegraphics[scale=1.5]{fig2}
	\caption{X and Y matrices (first rows only)}
	\end{figure}
		
		
	\begin{figure}[h!]
	\centering
	\includegraphics[scale=1.0]{fig3}
	\caption{ $\hat{\beta}$ and residuals}
	\end{figure}

	\begin{figure}[h!]
	\centering
	\includegraphics[scale=1.5]{fig4}
	\caption{Variance-covariance matrix and stderr}
	\end{figure}
				
	\begin{figure}[h!]
	\centering
	\includegraphics[scale=1.0]{fig5}
	\caption{t-test for $\hat{\beta_{0}}$ to $\hat{\beta_{8}}$ and p-values}
	\end{figure}
					
	\begin{figure}[h!]
	\centering
	\includegraphics[scale=1.0]{fig6}
	\caption{Sum of Squares, Mean Squares, and F sttaistics with p-value}
	\end{figure}
						
	\begin{figure}[h!]
	\centering
	\includegraphics[scale=1.5]{fig7}
	\caption{R Square, Adjusted R Square, and CV}
	\end{figure}						

	\begin{figure}[h!]
	\centering
	\includegraphics[scale=1.5]{fig8}
	\caption{Results using lm}
	\end{figure}
				
	\begin{figure}[h!]
	\centering
	\includegraphics[scale=1.5]{fig32a}
	\caption{Problem 3.2 A}
	\end{figure}
			
	\begin{figure}[h!]
	\centering
	\includegraphics[scale=1.5]{fig32b}
	\caption{Problem 3.2 B}
	\end{figure}			
			
	\begin{figure}[h!]
	\centering
	\includegraphics[scale=1.5]{fig32c}
	\caption{Problem 3.2 C}
	\end{figure}
				
	\begin{figure}[h!]
	\centering
	\includegraphics[scale=1.5]{fig34}
	\caption{Problem 3.4}
	\end{figure}	
					
%%%%%%%%%%%%%%%%%%%%%%%%%%%%%%%%%%%%%%%%%%%%%%%%%%%%%%%%%%%%%%%%%%%%%%%%%%%%%%%%%%%%%%%%%%%%%%%%%%%%%%%%%%%%%%%%%%%%%%%%%%%%%%%%%%%

\end{document} 
%%%%%%%%%%%%%%%%%%%%%%%%%%%%%%%%%%%%%%%%%%%%%%%%%%%%%%%%%%%%%%%%%%%%%%%%%%%%%%%%%%%%%%%%%%%%%%%%%%%%%%%%%%%%%%%%%%%%%%%%%%%%%%%%%%%
